\section{Экономическая часть}
\subsection{Введение}

Учитывая строгие требования к разработке при ограниченных ресурсах, следует
подчеркнуть, что экономическая эффективность является важным условием при создании
научно-технической продукции и занимает значимое место в дипломной работе. В данном
разделе проводятся расчеты затрат на выполнение научно-исследовательских задач, а
также демонстрируются знания, навыки и умения выпускников инженерных направлений в
области экономики и управления.

В специальной части дипломного проекта осуществляется разработка программного
обеспечения для расчета и проектирования оптимальной геометрии сопла жидкостного
ракетного двигателя, с учетом изменения свойств продуктов сгорания вдоль тракта сопла.

В конструкторской части дипломного проекта осуществляется выбор топливных компонентов
и разработка принципиальной схемы двигательной установки. В соответствии с
требованиями технического задания производится расчет основных параметров двигателя,
подбор материалов, выполнение прочностного расчета камеры сгорания, а также
составление технологического описания процесса сборки турбонасосного агрегата.

Технологическая часть дипломного проекта включает детальное рассмотрение процесса
сборки одного из турбонасосных агрегатов, с акцентом на особенности технологического
процесса.

В разделе, касающемся обеспечения безопасности производственного процесса
осуществляется оценка вредных и опасных факторов производства, способных повлиять на
жизнь и здоровье человека, вовлеченного в процесс сборки турбонасосного агрегата.

Организационно-экономический раздел дипломного проекта направлен на расчет сметы
затрат для разработки жидкостного ракетного двигателя первой ступени, использующего
топливо на основе кислорода и водорода.

