\subsection{Экономический подраздел}

В экономическом подразделе проводится расчет стоимости проекта. Все затраты в ходе выполнения проекта учитываются по рыночному принципу, как в случае выполнения стороннему заказчику по договору.

\subsubsection{Расчет себестоимости}\label{sec:Расчет себестоимости} % (fold)

Расходы, рассматриваемые в проекте, делятся на две группы:
\begin{itemize}
	\item Производственные расходы, включающие в себя материальные расходы, расходы на оплату труда, амортизационные отчисления и прочие расходы;
	\item Накладные расходы.
\end{itemize}

К материальным расходам относятся затраты на сырье, материалы, топливо, воду, комплектующие, а также на не амортизируемые имущества (стоимостью менее 100 000 рублей без НДС или сроком полезного использования 1 года) и др. Перечень расходов, относящихся к материальным представлены в виде таблицы~\ref{tab:mat-costs}. Расходы на оплату труда приводятся в виде таблице~\ref{tab:salary}.

\begin{table}
	\caption{Материальные расходы}\label{tab:mat-costs}
	\begin{tabularx}{\textwidth}{|l|c|p{2.78cm}|X|}
		\hline Объект                      & Количество & Цена единицы, руб. & Общая стоимость, руб.            \\
		\hline Персональный компьютер      & 2          & \centering 40000   & \centering\arraybackslash 80000  \\
		\hline Рабочий стол                & 2          & \centering 5000    & \centering\arraybackslash 10000  \\
		\hline Кресло                      & 2          & \centering 6000    & \centering\arraybackslash 12000  \\
		\hline МФУ                         & 1          & \centering 15000   & \centering\arraybackslash 15000  \\
		\hline Канцелярские принадлежности & 2          & \centering 1000    & \centering\arraybackslash 2000   \\
		\hline Microsoft Office 365        & 2          & \centering 1500    & \centering\arraybackslash 3000   \\
		\hline Компас-3D                   & 1          & \centering 32500   & \centering\arraybackslash 32500  \\
		\hline Итого материальные расходы  &            &                    & \centering\arraybackslash 154500 \\
		\hline
	\end{tabularx}
\end{table}

\begin{table}[H]
	\caption{Расчет затрат на заработную плату}\label{tab:salary}
	\begin{tabularx}{\textwidth}{|l|p{3cm}|p{3cm}|X|}
		\hline Должность             & Заработная плата за месяц, руб. & Фактически отработанное время,  дни & Заработная плата за срок выполнения проекта, руб. \\
		\hline Главный-конструктор   & \centering 120000 (0,5 ставки)  & \centering 53                       & \centering\arraybackslash 302000                  \\
		\hline Инженер-проектировщик & \centering 70000                & \centering 42                       & \centering\arraybackslash 140000                  \\
		\hline Инженер-проектировщик & \centering 70000                & \centering 21                       & \centering\arraybackslash 70000                   \\
		\hline Инженер-конструктор   & \centering 70000                & \centering 16                       & \centering\arraybackslash 53333                   \\
		\hline Инженер-технолог      & \centering 80000                & \centering 10                       & \centering\arraybackslash 38095                   \\
		\hline Итого                 &                                 &                                     & \centering\arraybackslash 603428                  \\
		\hline
	\end{tabularx}
\end{table}

Таким образом рассчитываются расходы на оплату труда только
руководителю проекта и исполнителям. Расходы на оплату труда
административно-управленческому персоналу будут учтены в
составе накладных расходов.

Амортизационных отчислений лицензионного программного
обеспечения Microsoft Office 365 и Компас-3D нет, так как
стоимость вышесказанного не превышает 100 000 рублей.

В состав прочих расходов относятся налоги, страховые взносы,
плата за аренду оборудования и другие расходы. Предприятие
обязано уплатить страховые взносы в размере 30 \% от фонда
оплаты труда (ФОТ) и страховые взносы на обязательное
социальное страхование от несчастных случаев на производстве
и профессиональных заболеваний от ФОТ в соответствии с видами
экономической деятельности по классам профессионального риска.

Научные исследования и разработки в области естественных и
технических наук – 1 класс профессионального риска (0,2 \%).

Прочие расходы приводятся в таблице~\ref{tab:other-expenses}.
Определение себестоимости продукта проводится по таблице~\ref{tab:cost-price}.

\begin{table}
	\caption{Расчет затрат на заработную плату}\label{tab:other-expenses}
	\begin{tabularx}{\textwidth}{|X|c|}
		\hline Статья расходов                                       & Общая сумма затрат, руб.    \\
		\hline Страховые расходы в размере 30\%                      & $603428 \cdot 0,3 = 181028$ \\
		\hline Расходы за 1-ый класс профессионального риска (0,2\%) & $603428 \cdot 0,002 = 1206$ \\
		\hline Итого                                                 & 182234                      \\
		\hline
	\end{tabularx}
\end{table}

\begin{table}
	\caption{Расчет затрат на заработную плату}\label{tab:cost-price}
	\begin{tabularx}{\textwidth}{|X|c|}
		\hline Наименование статьи затрат          & Сумма затрат, руб. \\
		\hline Прямые производственные затраты     & 1087662            \\
		\hline 1. Материальны расходы              & 302000             \\
		\hline 2. Расходы на заработную плату      & 603428             \\
		\hline 3. Амортизация                      & —                  \\
		\hline 4. Прочие расходы                   & 182234             \\
		\hline 4.1. Страховые взносы               & 182234             \\
		\hline 4.2. Расходы на аренду помещения    & —                  \\
		\hline 4.3. Расходы на аренду оборудования & —                  \\
		\hline Накладные расходы                   & 141396             \\
		\hline Итого                               & 1229058            \\
		\hline
	\end{tabularx}
\end{table}

Производственные расходы расчитываем по формуле
\begin{equation}
	\text{Пр} = 302000 + 603428 + 182234 = 1087662 \text{ рубля}.
\end{equation}

В организационно-экономическом разделе ВКР величина
накладных расходов принимается условно равной 13\% от
величины прямых расходов
\begin{equation}
	\text{Нр} = 1087662 \cdot 0.13 = 141396 \text{ рублей}.
\end{equation}

Итого (производственные расходы и накладные расходы)
\begin{equation}
	\text{Ир} = 1087662 + 141396 = 1229058 \text{ рублей}.
\end{equation}
% subsubsection Расчет себестоимости (end)

\subsubsection{Расчет прибыли}\label{sec:Расчет прибыли} % (fold)

Прибыль назначается в размере 20\% от размера итоговой суммы
производственных и накладных расходов
\begin{equation}
	\text{Прибыль} = 1229058 \cdot 0.2 = 254811 \text{ рублей}.
\end{equation}

Суммарная рыночная стоимость с учетом заложенной прибыли
\begin{equation}
	\text{Рыночная стоимость} = 1229058 + 254811 = 1474869 \text{ рублей}.
\end{equation}

Рыночная стоимость выполненной в рамках дипломного проекта
работы, рассчитанная затратным способом представлена в виде
круговой диаграммы, отражающей всю структуру стоимости ВКР
(Рис.~\ref{fig:project-market-value-chart}).

\begin{figure}[t]
	\centering
	\begin{tikzpicture}
		\pie{41/Расходы на ЗП, 20/Материальные расходы, 17/Прибыль, 12/Прочие расходы, 10/Накладные расходы}
	\end{tikzpicture}
	\caption{Диаграмма рыночной стоимости проекта}\label{fig:project-market-value-chart}
\end{figure}

\textbf{Вывод:} в данном разделе дипломного проекта был
проведен расчет затрат на разработку двигателя первой
ступени, построен сетевой график работ, было найдено
необходимое количество сотрудников и необходимое количество
рабочих дней для выполнения поставленной задачи. Стоимость
проекта составила 1 474 869 рублей.

% subsubsection Расчет прибыли (end)
